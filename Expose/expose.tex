\documentclass[fontsize=11pt, paper=a4, parskip=half]{scrartcl}
\pagestyle{plain}

\usepackage[T1]{fontenc}
\usepackage[utf8]{inputenc}
\usepackage[ngerman]{babel}
\usepackage{url}
\usepackage{hyperref}
\usepackage{lipsum}

\setkomafont{disposition}{\normalcolor\bfseries}

% In den folgenden Zeilen {...} jeweils durch die tatsächliche personen-/exposébezogene Angaben ergänzen
\title{
	TITEL DES EXPOSÉS
}

\subtitle{Exposé zur Seminararbeit}

\author{
	Johannes Teklote
	\\
	\texttt{jotek001@stud.fh-dortmund.de}
	\\ \\
	{7091992}
}

\begin{document}

\maketitle

\section{Motivation}
Laut \cite{Balzert2009} entfallen 80\% des Aufwands in der Softwareentwicklung auf deren Wartung, wobei 40\% davon benötigt werden, um die zu wartende Software zu verstehen.
Ein erheblicher Teil der Kosten für die Softwareentwicklung entfällt damit auf das Verstehen der Software.
Wenn insgesamt die Entwicklungskosten reduziert werden sollen, stellt dieser Umstand einen wesentlichen Hebel dar.
Denn der Wartungsaufwand im Übrigen lässt sich vielfach nicht reduzieren, weil sich die Anforderungen an die Software im Laufe der Zeit ändern, weil auftretende Probleme behoben werden müssen oder weil Abhängigkeiten auf andere Software, Frameworks oder Komponenten angepasst werden müssen.
Durch das erfüllen gewisser Qualitätsstandards lässt sich bereits der für das Verstehen des Codes notwendige Aufwand reduzieren.

Qualitativ hochwertige Software zeichnet sich nach \cite{ISO-25010} dadurch aus, dass sie stabil ist und gut gewartet, analysiert, erlernt und verändert werden kann.
Software, die diese Anforderungen erfüllt, beinhaltet Code, der nach bestimmten Richtlinien gebildet wird.
Voraussetzung ist jedoch, dass sich alle Entwickler eines Teams auch an diese vorher festgelegten und für die gesamte Projektlaufzeit geltenden Richtlinien halten.
Zu diesen Regeln gehört beispielsweise, dass der Code auf einer Ebene einheitlich eingerückt wird, dass öffnende geschweifte Klammern in der gleichen Zeile stehen, wie der Code, der die Klammer erfordert, dass Variablennamen sprechend sind und nicht nur aus einem Buchstaben bestehen oder dass Parameter auf ihre Gültigkeit überprüft werden, bevor mit ihnen gearbeitet wird.

Die Einhaltung solcher Regeln scheint auf den ersten Blick nur von geringer Bedeutung zu sein, allerdings lenken beispielsweise Unstimmigkeiten in der Formatierung beim Lesen des Codes ab und sie erschweren es, sich auf den eigentlichen Code zu konzentrieren.
In \cite{PJ2015} und \cite{SP2011} werden diese Formatverstöße als Hintergrundrauschen beschrieben, das vom eigentlichen Code ablenkt. 

\section{Problemstellung}
Code, der nach den oben beschriebenen Regeln geschrieben wurde, ist laut \cite{PJ2015} leichter verständlich und dadurch auch leichter erlernbar und veränderbar.
Um die Teammitglieder dazu zu bringen, sich an die vereinbarten Regeln zu halten, gibt es zwei Möglichkeiten.
Zum einen kann im Reviewprozess des Projekts ein automatischer Test eingebaut werden, der dafür sorgt, dass grundsätzlich funktionierender Code dann nicht angenommen wird, wenn er gegen festgelegte Regeln verstößt, wenn er also beispielsweise Stylingrichtlinien nicht einhält oder es potentiell unsichere Variablenzugriffe gibt.
Zum anderen können die Teammitglieder durch Belohnungen dazu motiviert werden, von sich aus die festgelegten Regeln einzuhalten und dies auch selbst zu überprüfen. 
Um eine solche Motivation zu erreichen, kann Gamification verwendet werden.
Hierbei werden Methoden genutzt, die aus Computerspielen bekannt sind.
Hierzu gehören beispielsweise Levels, Badges oder Leader Boards, die das Team oder das einzelne Teammitglied erreichen kann. 

Beide Verfahren führen dazu, dass die Teammitglieder die vereinbarten Programmierrichtlinien mit der Zeit von sich aus einhalten, ohne immer darauf hingewiesen werden zu müssen, entweder, weil sie gelernt haben, dass der eingereichte Code sonst aus Formgründen abgelehnt wird oder weil sie für besseren Code beispielsweise mehr Erfahrungspunkte bekommen.
Es entsteht also ein Lerneffekt, der bewirkt, dass die Teilnehmer auch in Zukunft qualitativ hochwertigen Code einreichen.
In beiden Fällen wird die eigene Motivation, guten Code zu schreiben, durch eine extrinsische Motivation verstärkt.
Im Rahmen dieser Arbeit liegt der Fokus auf der Motivation durch Gamification.


\section{Zielsetzung}
Ziel dieser Arbeit ist die Erweiterung der Codeanalyseplattform coderadar um Gamification-Elemente zur Steigerung der Code-Qualität.
Dazu werden zunächst ausgewählte Gamification-Elemente in coderadar implementiert.
Anschließend wird eine Balancierungsmatrix für eine gleichmäßige Verteilung der Punkte, auf denen Levels, Leader Boards und ähnliches basieren, erarbeitet.
Abschließend wird ein Feldversuch durchgeführt, um festzustellen, in wie weit die vorgestellten Maßnahmen die Softwarequalität nachhaltig verbessern konnten.

\section{Vorgehensweise}
Zunächst werden die Grundlagen dieser Arbeit vorgestellt.
Es wird erläutert, was Gamification ist und welche Gamification-Elemente in dieser Arbeit Verwendung finden werden.
Daneben wird im Rahmen der Vorstellung verwandter Arbeiten darauf eingegangen, welche Implementierungen und Implementierungsansätze es bereits für Gamification-Elemente in Zusammenhang mit der Code-Qualität gibt.
Außerdem wird die Codeanalyseplattform coderadar vorgestellt, die als Basis für die im Rahmen dieser Arbeit erfolgende Implementierung dienen wird.

Im Anschluss daran wird das Spielkonzept vorgestellt.
Dazu gehört zunächst die Auswahl der Metriken, anhand derer die Codequalität bewertet wird.
Entsprechend diesen Metriken wird der konkrete Code dann untersucht.
Anhand des Ergebnisses dieser Untersuchung werden anschließend die Punkte verteilt, auf denen später Levels oder Leader Boards beruhen.
Des Weiteren werden die Spielmechanismen, die Spielregeln und das Spielziel erläutert.
Außerdem wird eine Balancierungsmatrix ausgearbeitet, die für eine gleichmäßig gewichtete Punkteverteilung für unterschiedliche Aktionen sorgt.
Sie bewirkt beispielsweise, dass umfangreiche Änderungen verhältnismäßig mehr Punkte bringen als weniger umfangreiche Änderungen.

Danach wird auf die Implementierung eingegangen.
Zunächst werden dafür die Anforderungen erhoben und eine Definition of Done, also die Definition, wann das Projekt fertig ist, erstellt.
Auf Basis dieser Anforderungen werden dann die erforderlichen Änderungen an der coderadar-Plattform modelliert.
Dann erfolgt die eigentliche Implementierung.

Anschließend erfolgt die Evaluierung.
Dazu wird zunächst ein Projekt zur Untersuchung in coderadar eingecheckt und anhand der ausgewählten Metriken wird die Code-Qulaität bewertet.
Der hier ermittelte Wert soll als Grundlage für die Messung der Bewertungsänderung dienen.
Über eine gewisse Zeit wird das Projekt dann im Rahmen eines Feldversuchs verfolgt und es wird beobachtet, in wie weit sich die Bewertung des untersuchten Projektes verändert.
Zusätzlich dazu sollen die Projektteammitglieder hinsichtlich der Arbeit mit den Gamification-Elementen in coderadar befragt werden.

Abschließend wird das Ergebnis der Arbeit zusammengefasst.
Auf Basis der Bewertungsveränderung und der Ergebnisse der Umfrage wird ein Fazit zur vorgestellten Implementierung von Gamification-Elementen und ihrem Einfluss auf nachhaltige Code-Qualität gezogen.
Als Grundlage für dieses Fazit dienen dabei zum Einen die Beobachtung über die Veränderung der Bewertung der Code-Qualität und zum Anderen die Ergebnisse aus der Befragung der Projektteammitglieder.
Außerdem wird reflektiert, welche Herausforderungen es bei der Implementierung und der Durchführung des Feldversuchs gab.
Zum Schluss wird ein Ausblick auf die weitere Arbeit gegeben.

\pagebreak

\bibliographystyle{alphadin}
\bibliography{literature}
\end{document}
